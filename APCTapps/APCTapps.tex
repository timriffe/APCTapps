\documentclass[11pt,oneside]{article} % for sharing

\usepackage{amsmath}
\usepackage{array}
\usepackage{caption}
\usepackage{placeins}
\usepackage{graphicx}
\usepackage{subcaption}
\usepackage{longtable}
\usepackage{setspace}
\usepackage{booktabs}
\usepackage{chngpage}
\usepackage{natbib}
\bibpunct{(}{)}{,}{a}{}{;} 
\usepackage{url}
\usepackage{nth}
\usepackage{authblk}

% end preamble
%-------------------------------------------

\begin{document}

\title{Domains of application for the demographic time framework}

\author[1]{Tim Riffe\thanks{riffe@demogr.mpg.de}}
\affil[1]{Max Planck Institute for Demographic Research}


%\author{[Authors]}


\maketitle

\begin{abstract}
The demographic time framework defines the time-space for birth-death processes
that produce overlapping lifecourses. Although its original description was
motivated by the study of health patterns over the human lifecourse, where complete human lifespans define the space boundary, this need not be the case.
In fact, the framework is generally applicable to any domain where
population-level variation occurs over time and within ``life''. Since the
framework is an abstraction, both intuitively coherent, but not
necessarily obvious to translate into applications, we here offer a potluck of
suggested applications in a variety of domains, both within and outside demography.
\end{abstract}

\section*{Background}

%Since change and process can only unfold in time, time is essential to all
%scientific inquiry. That events and change can be placed on a timeline, whether
% explicit chain reactions, a gradual unfolding of argumentation, a trajectory of growth or metabolism, or a population mean level
%of some health metric, 

%In many applications of human demography, time-to-event is inferred rather than
%measured directly

A singular relationship unites six differemt measures of demographic time,
chronological age (A), period (P), birth cohort (C), thanatological age (T), death cohort (D), and
lifespan (L). Together, these time measures define a time-space in three
dimensions. Further, there are four triad-identities within this set of six, and
each of these tesselates to form a plane, or characteristic diagram.

\section{Applications}

\section{Human demography}
	\subsection{Reproduction}
		\subsubsection{Birth spacing}
		Time since birth i, time until birth i+1, interval (TAL)
		Time since birth i, year of birth i, year (APC)
		Time until birth i+1, year, year of birth i+1 (TPD)
		interval, year of birth i, year of birth i+1 (LCD)
		
		Note that this does not include age, but it captures the interval period as a
		complete space.
		\subsubsection{In-utero development}
		time since conception, time until parturition, and duration of gestation
		together combine into a TAL identity that also lends itself to exploration of
		patterns of fetal development, placenta development, and maternal health.
	\subsection{Health}
		\subsubsection{Prevalence-based measures}
		HRS study
		\subsubsection{Incidence-based measures}
		time-to-death patterns of prevalence mechanically be produced by
		chronological-age-patterned incidence rates. Specifically, if a sucession of
		health states triggered by a single chronological-age-patterned onset itself
		implies increasing probabilities of death, this may produce a time-to-death
		pattern of prevalence. Whether to model using prevalence or incidence, is
		however a tradeoff of convenience.
	\subsection{Migration}
	\subsection{Labor force demography}
	\subsection{Special subpopulations}
		\subsubsection{Professors}
		\subsubsection{Prisoners}
		Data downloaded
		\subsubsection{Athletes}
		Here sumo or footballers, will see. Email sent to sumodb maintainer Sept 27,
		2016.
\section{Non-human demography}	
	\subsection{Flora}
	\subsection{Fauna}
	\subsection{Natural phenomena}
	\subsection{Manmade objects}
	
\section{Discussion}

Probably worth talking about statistical inference here.

\section{Conclusions}





\cite{*}


\singlespacing
\bibliographystyle{plainnat}
  \bibliography{references} 

\end{document}