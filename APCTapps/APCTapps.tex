\documentclass[11pt,oneside]{article} % for sharing

\usepackage{amsmath}
\usepackage{array}
\usepackage{caption}
\usepackage{placeins}
\usepackage{graphicx}
\usepackage{subcaption}
\usepackage{longtable}
\usepackage{setspace}
\usepackage{booktabs}
\usepackage{chngpage}
\usepackage[style=authoryear-comp]{natbib}
\bibpunct{(}{)}{,}{a}{}{;} 
\usepackage{url}
\usepackage{nth}
\usepackage{authblk}
\DeclareRobustCommand{\VAN}[3]{#2} 
% end preamble
%-------------------------------------------

\begin{document}

\title{Domains of application for the demographic time framework}

\author[1]{Tim Riffe\thanks{riffe@demogr.mpg.de\\ ~~~~(author order tbd)}}
\author[2,3]{Jonas Sch{\"o}ley}
\author[2,3]{Francisco Villavicencio}
\affil[1]{Max Planck Institute for Demographic Research}
\affil[2]{University of Southern Denmark}
\affil[3]{Max-Planck Odense Center on the Biodemography of Aging}

%\author{[Authors]}


\maketitle

\begin{abstract}
The demographic time framework defines the time-space for birth-death processes
that produce overlapping lifecourses. The framework is generally applicable to
any domain where population-level variation occurs over time and within ``life''. Since the
framework is an abstraction, both intuitively coherent, but not
necessarily obvious to translate into applications, we here offer a potluck of
suggested applications in a variety of domains, both within and outside demography.
Application domains include subfields of human demography focusing on particular transitions, stages of the life course, or special sub populations (prisoners, athletes, and professors), but also applications to non-human species, natural phenomena, and artificial objects.
\end{abstract}

\section*{Background}

%Since change and process can only unfold in time, time is essential to all
%scientific inquiry. That events and change can be placed on a timeline, whether
% explicit chain reactions, a gradual unfolding of argumentation, a trajectory of growth or metabolism, or a population mean level
%of some health metric, 

%In many applications of human demography, time-to-event is inferred rather than
%measured directly

A singular relationship unites six distinct measures of demographic time:
chronological age (A), period (P), birth cohort (C), thanatological age (T,
also known as time-to-death), death cohort (D), and lifespan (L). Together,
these time measures define a time-space in three dimensions. The set of six
time measures also contains four three-member identities, each of which
tesselates to form a plane, or characteristic diagram. These identities are the
well-known APC identity, as well as the TPD, LCD, and TAL identities. A
fuller description about how these identities and diagrams relate, as well as an
example about how the time framework might guide scientific inquiry is given in
\citet{rsv2015}. 

Although its original description was
motivated by the study of health patterns over the human lifecourse, where
complete human lifespans define the lifecourse boundary, this need not be the
case.
The only example of an application given thus far has dealt with late-life
health patterns \citep{riffe2015ttd}, and in this example the entire human
lifespan set the space boundary. While this usage case is useful as a heuristic,
and for qualifying morbidity comparisons or guiding morbidity projections
\citep{vanRaalte2015HLE}, it is also clear that providing more
examples of applications would help illustrate the transportability of the
framework and help guide other researchers in exploring its uses. For this
reason, we here aim to provide as diverse-as-possible set of applications in
different domains of inquiry. The domains that we aim to cover range from
standard human demography, medical research, special sub-populations defined by
variable points of entry and exit, biology, the so-called hard sciences, and
engineering. With such a wide variety of application domains, this treatment
will be neccesarily brief with respect to each, in most cases consisting in
sketches of potential analyses that we think would be feasible, but that we do
not have the data or expertise to carry out rigorously ourselves. For others, we
work up some exploratory results given novel perspectives on data already
available. 

The empirical example we provide in this short abstract is based on rich
data produced about Major League baseball players in the United States
\citep{Lahman}, which is structured in such as way as to easily conform with the
demographic time framework. Specifically, this database provides not only birth
and death (where applicable) dates of major league players, but also season
statistics. For the exploratory analysis begin, we specifically look at
pitching statistics as a function of time since debuting and time until retiring
from professional play, thereby implying the third time dimension of career
length. This particular setup is an event-history translation of the TAL
diagram.

Further examples that will be included as this paper progresses are included in
outline format in what follows. These include novel time-perspectives on
demographic topics including birth-spacing, in-utero development, health
measures, migration, and labor force demography. Special population applications
will include examples from baseball player careers, recidivism or prison
populations, and professor populations. Applications in non-human demography
will come from bacterial growth and reproduction trajectories, 

All examples given here are decidedly visual in nature, and this only
represents the first step in a complete analysis: that of pattern detection for
phenomena.

At this point 

\section{Applications}

\section{Human demography}
	\subsection{Reproduction}
		\subsubsection{Birth spacing}
		Time since birth i, time until birth i+1, interval (TAL)
		Time since birth i, year of birth i, year (APC)
		Time until birth i+1, year, year of birth i+1 (TPD)
		interval, year of birth i, year of birth i+1 (LCD)
		
		Note that this does not include age, but it captures the interval period as a
		complete space.
		\subsubsection{In-utero development}
		time since conception, time until parturition, and duration of gestation
		together combine into a TAL identity that also lends itself to exploration of
		patterns of fetal development, placenta development, and maternal health.
	\subsection{Health}
		\subsubsection{Prevalence-based measures}
		HRS study
		\subsubsection{Incidence-based measures}
		time-to-death patterns of prevalence mechanically be produced by
		chronological-age-patterned incidence rates. Specifically, if a sucession of
		health states triggered by a single chronological-age-patterned onset itself
		implies increasing probabilities of death, this may produce a time-to-death
		pattern of prevalence. Whether to model using prevalence or incidence, is
		however a tradeoff of convenience.
	\subsection{Migration}
	\subsection{Labor force demography}
	Time measures
	\subsection{Special subpopulations}
		\subsubsection{Professors}
		\subsubsection{Prisoners}
		Data downloaded \citep{LSVCB}
		\subsubsection{Athletes}
		Here sumo or footballers, will see. Email sent to sumodb maintainer Sept 27,
		2016.
\section{Non-human demography}	
	\subsection{Bacterial growth}
	\subsection{Natural phenomena}
	\subsection{Manmade objects}
	
\section{Discussion}

Probably worth talking about statistical inference here.

\section{Conclusions}


\singlespacing
\bibliographystyle{plainnat}
  \bibliography{references} 

\end{document}